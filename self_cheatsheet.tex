% Copyright (c) 2016 Bystroushaak

% Permission is hereby granted, free of charge, to any person
% obtaining a copy of this software and associated documentation
% files (the "Software"), to deal in the Software without
% restriction, including without limitation the rights to use,
% copy, modify, merge, publish, distribute, sublicense, and/or sell
% copies of the Software, and to permit persons to whom the
% Software is furnished to do so, subject to the following
% conditions:

% The above copyright notice and this permission notice shall be
% included in all copies or substantial portions of the Software.

% THE SOFTWARE IS PROVIDED "AS IS", WITHOUT WARRANTY OF ANY KIND,
% EXPRESS OR IMPLIED, INCLUDING BUT NOT LIMITED TO THE WARRANTIES
% OF MERCHANTABILITY, FITNESS FOR A PARTICULAR PURPOSE AND
% NONINFRINGEMENT. IN NO EVENT SHALL THE AUTHORS OR COPYRIGHT
% HOLDERS BE LIABLE FOR ANY CLAIM, DAMAGES OR OTHER LIABILITY,
% WHETHER IN AN ACTION OF CONTRACT, TORT OR OTHERWISE, ARISING
% FROM, OUT OF OR IN CONNECTION WITH THE SOFTWARE OR THE USE OR
% OTHER DEALINGS IN THE SOFTWARE.

% Inspired by http://tex.stackexchange.com/questions/99765/document-class-for-reference-cards
\documentclass[10pt]{article}
\usepackage{fixltx2e}
\usepackage[orthodox,l2tabu,abort]{nag}
\usepackage{textcomp}
\usepackage{hyperref}
\usepackage{dirtree}
% \setlength{\DTbaselineskip}{5pt}
\DTsetlength{0.2em}{0.5em}{0.3em}{0.4pt}{1.6pt}

% Page layout
\usepackage[a4paper,landscape,margin=1cm]{geometry}

\usepackage{multicol}
\setlength{\columnsep}{0.5cm}


% Title area
\usepackage{titling} % Allows for use of date, author, etc. after \maketitle
% Ref: http://tex.stackexchange.com/questions/3988/titlesec-versus-titling-mangling-thetitle
\let\oldtitle\title
\renewcommand{\title}[1]{\oldtitle{#1}\newcommand{\mythetitle}{#1}}
\renewcommand{\maketitle}{
    {\begin{center}\Large \mythetitle\end{center}}
}

% Document divisions
\usepackage{titlesec}
\setcounter{secnumdepth}{0}
\titlespacing{\section}{0pt}{0pt}{0pt}
\titlespacing{\subsection}{0pt}{3pt}{0pt}
\titlespacing{\subsubsection}{0pt}{0pt}{0pt}
\setlength{\parskip}{2pt}

\usepackage{nopageno} % To keep \section from resetting page style
\setlength{\parindent}{0pt} % disabling indentation by default

% Lists
\usepackage{enumitem} % for consistent formatting of lists
\newlist{ttdesc}{description}{1}
\setlist[ttdesc]{font=\ttfamily,noitemsep}
\usepackage{calc} % for \widthof

\usepackage{lipsum}

% Metadata definition
\title{Selflang cheat sheet}
\author{Bystroushaak}
\date{2016}

\pdfinfo {
    /Author   (\theauthor)
    /Title    (\thetitle)
    /Subject  (\thetitle)
    /Keywords (Self, SelfLang, programming language, programming)
}


% Set listing
% https://en.wikibooks.org/wiki/LaTeX/Source_Code_Listings
\usepackage{listings}
\usepackage{pxfonts}

\lstset{
    basicstyle=\ttfamily,
    keywordstyle=\bfseries,
    showstringspaces=false,
    breaklines=false,
    breakatwhitespace=false,
    columns=fullflexible,
}


% Content =====================================================================
\begin{document}
\begin{multicols*}{3}
\maketitle

\section{Object syntax}

An \textit{object} consists from (possibly empty) set of \textit{slots} and (optionally) \textit{code}.

Slots behave as key-value lookup table for data or objects. \textit{Code} is a sequence of \textit{expressions} (\textit{message sends} and \textit{literals}) separated by dots, and evaluated in order.

Syntax is straightforward:

\begin{lstlisting}
(| slot1. slot2 | 'code' printLine. 'str')
\end{lstlisting}

for object with \textit{slots} and \textit{code}, or

\begin{lstlisting}
(| | 'this is str') or ('this is str')
\end{lstlisting}

for object with \textit{code}, but no \textit{slots}, or

\begin{lstlisting}
(| slot1. slot2. slotN |)
\end{lstlisting}

for object with \textit{slots}, but no \textit{code}, or

\begin{lstlisting}
(| |) or ()
\end{lstlisting}

for empty object.



\subsection{Comments}

Comments use double quotes and are ignored by parser. See \textit{Annotation slot} in subsection \textit{Special slots}.

\begin{lstlisting}
("empty object same as () or (| |)")
\end{lstlisting}




\subsection{Slot assignments}

Slots can be assigned at the definition, either read only using \texttt{=} operator:

\begin{lstlisting}
(| slot = 1 |)
\end{lstlisting}

or rewritable using \texttt{<-} operator: % TODO: operator, really?

\begin{lstlisting}
(| slot <- 1 |)
\end{lstlisting}




\subsection{Messages}

\textit{Messages} are evaluated from left to right. Object is on the left, messages on the right. Messages can be without arguments:

\begin{lstlisting}
obj message
\end{lstlisting}

(to the \textit{obj} send a \textit{message}), or with arguments:

\begin{lstlisting}
obj message: argument
\end{lstlisting}

(to the \textit{obj} send a \textit{message} with \textit{argument} as value).




\subsection{Special slots}

\subsubsection{\texttt{self} slot}
Every object has automatically created slot named \texttt{self}, pointing to object itself. Unlike other languages, \texttt{self} may be omitted, in \textit{message sends}, so

\begin{lstlisting}
self message
\end{lstlisting}

is same thing as

\begin{lstlisting}
message
\end{lstlisting}


\subsubsection{Annotation slot}

Additional informations may be provided in the \textit{annotation slot}:

\begin{lstlisting}
(| {} = 'Annotation string.' |)
\end{lstlisting}

or

\begin{lstlisting}
(| { 'Annotation string.' slot. another. } |)
\end{lstlisting}


\subsubsection{\texttt{parent*} slot}
Every object may also contain \texttt{parent*} slot (star is mandatory), to which the lookups unresolved in \texttt{self} slot is delegated. This implements the inheritance.




\subsection{Method slots}

\textit{Method slots} is the storage for the code in objects.

\subsubsection{Unary slots}
Code \textit{slots} without arguments are called \textit{unary}.

\begin{lstlisting}
(| first = (printLine) |)
\end{lstlisting}

This \textit{method slot} can be invoked by sending \textit{unary} message:

\begin{lstlisting}
obj first
\end{lstlisting}



\subsubsection{Binary slots}

Slots with one argument is called \textit{binary}:

\begin{lstlisting}
(| first: = (| :arg | arg printLine) |)
\end{lstlisting}

Which has shorter equivalent:

\begin{lstlisting}
(| first: arg = (arg printLine) |)
\end{lstlisting}

Invocation is possible using the \textit{binary message}:

\begin{lstlisting}
obj first: 1
\end{lstlisting}

\textit{Binary slots} are also used as operators (one or more characters from \texttt{!@\#\$\%\^{}\&*-+=\textasciitilde/?\textless\textgreater,;|'\textbackslash{}} set).



\subsubsection{Keyword slots}

It is also possible to create \textit{keyword} slots with multiple arguments:

\begin{lstlisting}
(| first: x Second: y = (x + y printLine.) |)
\end{lstlisting}

Notice the upper-case \textit{S} in \texttt{Second}. Invocation is done via \textit{keyword message}:

\begin{lstlisting}
obj first: 3 Second: 5
\end{lstlisting}



\subsubsection{Priorities}

Constant definition \textgreater{} Unary \textgreater{} Binary \textgreater{} Keyword messages. Constant = self \textbar{} number \textbar{} string \textbar{} object.




\subsection{Block objects \textit{(closures)}}

\textit{Block objects} are Self closures, which means, that \texttt{self} slot is not present. Unresolved lookups are delegated via \texttt{parent*} to namespace surrounding the \textit{block object} at the time of creation. Syntax is same as with objects, but square brackets are used:

\begin{lstlisting}
[| slot1. slot2 | 'code']
\end{lstlisting}


Another difference is support of \textit{non-local return statement} using \texttt{\^} preceding the returned value. This returns the value not just from the \textit{closure}, but also from the enclosing namespace.

\begin{lstlisting}
(|
    x = 6.
    non_zero = (x > 5 ifTrue: [ ^ 'ok' ]
                         False: [ ^ 'nope' ])
|)
\end{lstlisting}

\^{} doesn't just return from the closure in \texttt{[]}, but also from the surrounding \textit{method object}. \texttt{non\_zero} message thus returns \texttt{'ok'}.



\subsubsection{Block messages}

\textit{Block objects} take by default message \texttt{value}, which evaluates the code and returns the value.

If the \textit{block object} takes (multiple) parameters, they may be supplied using \texttt{value:\ With:} message pattern:

\begin{lstlisting}
block_obj value: x With: y .. With: z
\end{lstlisting}

Notice the upper-case first letters in \texttt{With:} message. This signals that this is still part of one message with multiple arguments. In case of blocks, \textbf{unrequested arguments are ignored}, so you may use unlimited number of \texttt{With:}.




%%%%%%%%%%%%%%%%%%%%%%%%%%%%%%%%%%%%%%%%%%%%%%%%%%%%%%%%%%%%%%%%%%%%%%%%%%%%%%%
\pagebreak %%%%%%%%%%%%%%%%%%%%%%%%%%%%%%%%%%%%%%%%%%%%%%%%%%%%%%%%%%%%%%%%%%%%
\section{Namespaces}

\subsection{Traits}
Objects with shared behaviour, which is used by pointing \texttt{parent*} slots to them. This is analogical to \textit{sub-classing} in other languages.



\subsection{Mixins}

Small objects filled with shared code. This code doesn't contain \texttt{parent*} slot, and it is thus not directly part of the OOP tree. Approximate analogy to other languages would be \textit{interface with implementation}.



\subsection{Globals}
Prototype objects and \textit{oddballs} (unique singletons) like \texttt{true}, \texttt{false}, \texttt{nil}..




\section{Resends}

\textit{Resends} allows you to delegate the messages to the parent branches of OOP tree. Resend messages are equivalent of \textit{super} calls in other languages.

Syntactically, \textit{resends} are implemented using \textit{resend.} prefix for resent messages:

\begin{lstlisting}
resend.unary
resend.+ 1
resend.keyword: 1 Another: 2
\end{lstlisting}

You may also use \textit{directed resends}, targeting specific parent:

\begin{lstlisting}
intParent.+ 1
otherParent.keyword: 1 Another: 2
\end{lstlisting}




\section{Mirrors}
\textit{Mirrors} provide Self with introspection capabilities.

\textit{Mirror} can be created by sending \texttt{reflect:\ x} message to any \texttt{defaultBehavior} instance. The message will return \textit{dictionary-like} \textit{mirror} object.

\begin{lstlisting}
defaultBehavior reflect: 1
\end{lstlisting}

\textit{Mirror} presents you with sort of introspection layer usable for structural changes in \textit{reflectee} (original object \texttt{x}) by manipulating the \textit{dictionary-like} object.

This may be used for examination, addition or removal of the \textit{slots} of the \textit{reflectee}.




\vfill
\columnbreak
\section{Collections}
Various containers for data are implemented as \textit{key:\ val} storages. Even lists use this convention (elements are used both as key and value).

Self offers a rich variations of \textit{Sets}, \textit{Dictionaries} and \textit{Trees}:

\vspace*{0.4cm}
% http://texblog.org/2012/08/07/semi-automatic-directory-tree-in-latex/
\dirtree{%
 .1 traits \textbf{collection}.
 .2 traits \textbf{abstractSetOrDictionary}.
 .3 traits \textbf{abstractSet}.
 .4 traits \textbf{universalSet}.
 .5 \textbf{universalSet}.
 .4 traits \textbf{hashTableSet}.
 .5 \textbf{identitySet} parent.
 .5 \textbf{customizableSet} parent.
 .5 \textbf{reflectiveIdentitySet} parent.
 .5 traits \textbf{set}.
 .5 traits \textbf{sharedSet}.
 .3 traits \textbf{abstractDictionary}.
 .4 traits \textbf{orderedDictionary}.
 .4 traits \textbf{universalDictionary}.
 .5 \textbf{universalDictionary}.
 .4 traits \textbf{hashTableDictionary}.
 .5 \textbf{identityDictionary} parent.
 .5 \textbf{customizableDictionary} parent.
 .5 \textbf{reflectiveIdentityDictionary} parent.
 .5 traits \textbf{dictionary}.
 .5 traits \textbf{sharedDictionary}.
 .2 traits \textbf{tree}.
 .3 traits \textbf{emptyTrees} abstract.
 .4 traits \textbf{emptyTrees bag}.
 .4 traits \textbf{emptyTrees set}.
 .3 traits \textbf{treeNodes} abstract.
 .4 traits \textbf{treeNodes bag}.
 .4 traits \textbf{treeNodes set}.
}

\vspace*{0.4cm}
\textit{Sets} behave like mathematical sets - unordered unique collection of values. \textit{Dictionaries} work as \textit{key: val} storages and are implemented using hashmaps.

\textit{Trees} are different implementations of \textit{dictionaries} using \textit{unbalanced binary trees}.

Note: If the elements are added in sorted order, trees may degenerate into lists, which may result in really bad performance.

\vfill
\columnbreak

There is also variety of \textit{Lists}, \textit{Vectors}, \textit{Strings} and \textit{Queues}:

\vspace*{0.4cm}
\dirtree{%
 .1 traits \textbf{collection}.
 .2 traits \textbf{path}.
 .2 traits \textbf{sharedQueue}.
 .2 traits \textbf{priorityQueue}.
 .2 traits \textbf{list}.
 .3 traits \textbf{orderedSet}.
 .3 \textbf{sortedList} parent.
 .4 \textbf{sortedListSet} parent.
 .2 traits \textbf{indexable}.
 .3 traits \textbf{mutableIndexable}.
 .4 traits \textbf{sequence}.
 .5 traits \textbf{sortedSequence}.
 .4 traits \textbf{vector}.
 .5 \textbf{vector}.
 .4 traits \textbf{byteVector}.
 .5 \textbf{byteVector}.
 .5 traits \textbf{int32or64}.
 .6 traits \textbf{int64}.
 .6 traits \textbf{int32}.
 .5 traits \textbf{string}.
 .6 traits \textbf{immutableString}.
 .7 traits \textbf{canonicalString}.
 .6 traits \textbf{mutableString}.
}

\vspace*{0.4cm}

Collections have rich message protocol, allowing various operations. Most important are:


\vspace*{0.2cm}
\small{\begin{tabular}{ p{50pt} p{180pt} l l }
Message & Description
\\\hline\hline

%%%%%%%%%%%%%%%%%%%%%%%%%%%%%%%%%%%%%%%%%%%%%%%%%%%%%%%%%%%%%%%%%%%%%%%%%%%%%%%
\texttt{at:}
&
Get item at position / key.
\\\hline %%%%%%%%%%%%%%%%%%%%%%%%%%%%%%%%%%%%%%%%%%%%%%%%%%%%%%%%%%%%%%%%%%%%%%

%%%%%%%%%%%%%%%%%%%%%%%%%%%%%%%%%%%%%%%%%%%%%%%%%%%%%%%%%%%%%%%%%%%%%%%%%%%%%%%
\texttt{at: Put:}
&
Put item to position / key.
\\\hline %%%%%%%%%%%%%%%%%%%%%%%%%%%%%%%%%%%%%%%%%%%%%%%%%%%%%%%%%%%%%%%%%%%%%%

%%%%%%%%%%%%%%%%%%%%%%%%%%%%%%%%%%%%%%%%%%%%%%%%%%%%%%%%%%%%%%%%%%%%%%%%%%%%%%%
\texttt{add:}
&
Add item (to the end in ordered) collections.
\\\hline %%%%%%%%%%%%%%%%%%%%%%%%%%%%%%%%%%%%%%%%%%%%%%%%%%%%%%%%%%%%%%%%%%%%%%

%%%%%%%%%%%%%%%%%%%%%%%%%%%%%%%%%%%%%%%%%%%%%%%%%%%%%%%%%%%%%%%%%%%%%%%%%%%%%%%
\texttt{addAll:}
&
Add all items to (end of) collection.
\\\hline %%%%%%%%%%%%%%%%%%%%%%%%%%%%%%%%%%%%%%%%%%%%%%%%%%%%%%%%%%%%%%%%%%%%%%

%%%%%%%%%%%%%%%%%%%%%%%%%%%%%%%%%%%%%%%%%%%%%%%%%%%%%%%%%%%%%%%%%%%%%%%%%%%%%%%
\texttt{do:}
&
Iterate over collections.
\\ %%%%%%%%%%%%%%%%%%%%%%%%%%%%%%%%%%%%%%%%%%%%%%%%%%%%%%%%%%%%%%%%%%%%%%%%%%%%
\end{tabular}}



\subsection{Collector}

\textit{Collector} is special kind of object created using \texttt{\&} operator. \textit{Collector} is not a collection, but can be converted to one.

Main reason to use it is the \texttt{\&} operator, which may simplify the syntax required to create such collection.

\nointerlineskip\begin{lstlisting}
(1 & ' and ' & 2) asList
\end{lstlisting}\nointerlineskip



\subsection{Point}
Another kind of container often used with \textit{collections} is \texttt{point} and naturally the \texttt{rectangle} made of two points.




\vfill
\columnbreak
\section{Control sequences}
As usual in \textit{Smalltalk-like} languages, \textit{control sequences} are implemented using message sends combined with block.




\subsection{Conditionals}
\textit{If condition} works by using messages defined in boolean (or boolean traits) objects:

\vspace*{0.2cm}

\small{\begin{tabular}{ p{70pt} p{140pt} l l }
Message & Description
\\ \hline %%%%%%%%%%%%%%%%%%%%%%%%%%%%%%%%%%%%%%%%%%%%%%%%%%%%%%%%%%%%%%%%%%%%%
\begin{lstlisting}
obj ifTrue: b
\end{lstlisting}
&\vspace*{0.25cm}
Execute \texttt{b} if the \textit{obj} is \texttt{true}.

\\ \hline %%%%%%%%%%%%%%%%%%%%%%%%%%%%%%%%%%%%%%%%%%%%%%%%%%%%%%%%%%%%%%%%%%%%%
\begin{lstlisting}
obj ifFalse: b
\end{lstlisting}
&\vspace*{0.25cm}
Execute \texttt{b} if the \textit{obj} is \texttt{false}.
\\\hline %%%%%%%%%%%%%%%%%%%%%%%%%%%%%%%%%%%%%%%%%%%%%%%%%%%%%%%%%%%%%%%%%%%%%%

%%%%%%%%%%%%%%%%%%%%%%%%%%%%%%%%%%%%%%%%%%%%%%%%%%%%%%%%%%%%%%%%%%%%%%%%%%%%%%%
\begin{lstlisting}
obj ifTrue: b1
     False: b2
\end{lstlisting}
&\vspace*{0.1cm}
\textit{b1} is executed, when \textit{obj} evaluates to \texttt{true}.
If not, (optional) \textit{b2} block is used.
\\\hline %%%%%%%%%%%%%%%%%%%%%%%%%%%%%%%%%%%%%%%%%%%%%%%%%%%%%%%%%%%%%%%%%%%%%%

%%%%%%%%%%%%%%%%%%%%%%%%%%%%%%%%%%%%%%%%%%%%%%%%%%%%%%%%%%%%%%%%%%%%%%%%%%%%%%%
\begin{lstlisting}
obj ifFalse: b1
     True: b2
\end{lstlisting}
&\vspace*{0.4cm}
Opposite of previous. 
\\ %%%%%%%%%%%%%%%%%%%%%%%%%%%%%%%%%%%%%%%%%%%%%%%%%%%%%%%%%%%%%%%%%%%%%%%%%%%%
\end{tabular}}




\subsection{Loops}
Looping is implemented as sending various loop messages to \textit{block objects}:

\vspace*{0.2cm}
\hrule
\vspace*{0.03cm}
\hrule

\begin{lstlisting}
[ ... ] loop
\end{lstlisting}
Loop over the \textit{block} indefinitely.

\vspace*{0.2cm}
\hrule

\begin{lstlisting}
[ proceed ] whileTrue: [ ... ]
\end{lstlisting}
Loop while \texttt{proceed} is \texttt{true}.

\vspace*{0.2cm}
\hrule

\begin{lstlisting}
[ quit ] whileFalse: [ ... ]
\end{lstlisting}
Loop while \texttt{quit} is \texttt{false}.

\vspace*{0.2cm}
\hrule

\begin{lstlisting}
[ ... ] untilTrue: [ quit ]
\end{lstlisting}
Loop until \texttt{quit} is \texttt{true}. Loop at least once.

\vspace*{0.2cm}
\hrule

\begin{lstlisting}
[ ... ] untilFalse: [ proceed ]
\end{lstlisting}
Loop until \texttt{proceed} is \texttt{false}. Loop at least once.

\vspace*{0.2cm}
\hrule

\begin{lstlisting}
[| :exit | ... cond ifTrue: exit ... ] loopExit
\end{lstlisting}
Loop until the \texttt{exit} parameter is not evaluated.

\vspace*{0.2cm}
\hrule

\begin{lstlisting}
[
    | :exit |
    ...
    cond ifTrue: [ exit value: expr ]
] loopExitValue
\end{lstlisting}
Loop until the \texttt{exit} parameter is not evaluated. Allows to return the \texttt{value} with the exit from the loop.

\vspace{0.2cm}
\hrule
\vspace{0.03cm}
\hrule

\subsubsection{integerIteration loops}

There is also loops defined as message sends to integers:

\begin{lstlisting}
numObj do: block
\end{lstlisting}

Do the \texttt{block} \textit{numObj} times.

\vspace*{0.2cm}
\hrule

\begin{lstlisting}
numObj to: end Do: block
\end{lstlisting}

Do the \texttt{block} each time counting from \textit{numObj} to \texttt{end}. For example \texttt{5\ to:\ 8\ Do:\ [|\ :i\ |\ i\ print]} will print \texttt{5678}.

Following messages are all variations of this message:

\begin{lstlisting}
numObj to: By: Do:
numObj to: ByNegative: Do:
numObj to: ByPositive: Do:
numObj upTo: Do:
numObj upTo: By: Do:
numObj downTo: Do:
numObj downTo: By: Do:
\end{lstlisting}




\section{Useful bits}
copy message

kde stáhnout

kde najít manuál

konference

% Footer
\vfill
\hrule
\smallskip

{\small
The cards may be freely distributed under
the terms of the MIT licence ---
Copyleft \textcopyleft\ \thedate{} \href{http://kitakitsune.org}{\theauthor}. Version 1.0.

\url{https://github.com/Bystroushaak/SelfCheatSheet}
}

\end{multicols*}
\end{document}
