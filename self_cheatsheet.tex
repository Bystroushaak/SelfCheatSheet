% Copyright (c) 2016 Bystroushaak

% Permission is hereby granted, free of charge, to any person
% obtaining a copy of this software and associated documentation
% files (the "Software"), to deal in the Software without
% restriction, including without limitation the rights to use,
% copy, modify, merge, publish, distribute, sublicense, and/or sell
% copies of the Software, and to permit persons to whom the
% Software is furnished to do so, subject to the following
% conditions:

% The above copyright notice and this permission notice shall be
% included in all copies or substantial portions of the Software.

% THE SOFTWARE IS PROVIDED "AS IS", WITHOUT WARRANTY OF ANY KIND,
% EXPRESS OR IMPLIED, INCLUDING BUT NOT LIMITED TO THE WARRANTIES
% OF MERCHANTABILITY, FITNESS FOR A PARTICULAR PURPOSE AND
% NONINFRINGEMENT. IN NO EVENT SHALL THE AUTHORS OR COPYRIGHT
% HOLDERS BE LIABLE FOR ANY CLAIM, DAMAGES OR OTHER LIABILITY,
% WHETHER IN AN ACTION OF CONTRACT, TORT OR OTHERWISE, ARISING
% FROM, OUT OF OR IN CONNECTION WITH THE SOFTWARE OR THE USE OR
% OTHER DEALINGS IN THE SOFTWARE.

% Inspired by http://tex.stackexchange.com/questions/99765/document-class-for-reference-cards
\documentclass[10pt]{article}
\usepackage{fixltx2e}
\usepackage[orthodox,l2tabu,abort]{nag}
\usepackage{textcomp}
\usepackage{hyperref}
\usepackage{dirtree}
\DTsetlength{0.2em}{0.5em}{0.3em}{0.4pt}{1.6pt}

% Page layout
\usepackage[a4paper,landscape,margin=1cm]{geometry}

\usepackage{multicol}
\setlength{\columnsep}{0.5cm}


% Title area
\usepackage{titling} % Allows for use of date, author, etc. after \maketitle
% Ref: http://tex.stackexchange.com/questions/3988/titlesec-versus-titling-mangling-thetitle
\let\oldtitle\title
\renewcommand{\title}[1]{\oldtitle{#1}\newcommand{\mythetitle}{#1}}
\renewcommand{\maketitle}{
    {\begin{center}\Large \mythetitle\end{center}}
}

% Document divisions
\usepackage{titlesec}
\setcounter{secnumdepth}{0}
\titlespacing{\section}{0pt}{0pt}{0pt}
\titlespacing{\subsection}{0pt}{3pt}{0pt}
\titlespacing{\subsubsection}{0pt}{0pt}{0pt}
\setlength{\parskip}{2pt}

\usepackage{nopageno} % To keep \section from resetting page style
\setlength{\parindent}{0pt} % disabling indentation by default

% Lists
\usepackage{enumitem} % for consistent formatting of lists
\newlist{ttdesc}{description}{1}
\setlist[ttdesc]{font=\ttfamily,noitemsep}
\usepackage{calc} % for \widthof

\usepackage{lipsum}

% Metadata definition
\title{Selflang cheat sheet}
\author{Bystroushaak}
\date{2016}

\pdfinfo {
    /Author   (\theauthor)
    /Title    (\thetitle)
    /Subject  (\thetitle)
    /Keywords (Self, SelfLang, programming language, programming)
}


% Set listing
% https://en.wikibooks.org/wiki/LaTeX/Source_Code_Listings
\usepackage{listings}
\usepackage{pxfonts}

\lstset{
    basicstyle=\ttfamily,
    keywordstyle=\bfseries,
    showstringspaces=false,
    breaklines=false,
    breakatwhitespace=false,
    columns=fullflexible,
}




% Content %%%%%%%%%%%%%%%%%%%%%%%%%%%%%%%%%%%%%%%%%%%%%%%%%%%%%%%%%%%%%%%%%%%%%
\begin{document}
\begin{multicols*}{3}
\maketitle




\section{Object syntax}
An \textit{object} consists from (possibly empty) set of \textit{slots} and (optionally) \textit{code}.

Slots behave as key-value lookup table, translating messages to objects. \textit{Code} is a sequence of \textit{expressions} (\textit{message sends} and \textit{literals}) separated by dots, evaluated in order.

Syntax is straightforward:

\begin{lstlisting}
(| slot1. slot2 | 'code' printLine. 'str')
\end{lstlisting}

for object with \textit{slots} and \textit{code}, or

\begin{lstlisting}
(| | 'this is str') or ('this is str')
\end{lstlisting}

for object with \textit{code}, but no \textit{slots}, or

\begin{lstlisting}
(| slot1. slot2. slotN |)
\end{lstlisting}

for object with \textit{slots}, but no \textit{code}, or

\begin{lstlisting}
(| |) or ()
\end{lstlisting}

for empty object.



\subsection{Comments}

Comments use double quotes and are ignored by parser.

\begin{lstlisting}
("empty object same as () or (| |)")
\end{lstlisting}



\subsection{Slot assignments}

Slots can be assigned at the time of definition, either read only using \texttt{=} operator:

\texttt{(| slot = 1 |)}

Or rewritable using \texttt{<-} operator. This operator actually adds two slots - \texttt{slot} for the data and \texttt{slot:} with assignment method object.

\texttt{(| slot <- 1 |)}



\subsection{Messages}

\textit{Messages} are evaluated from left to right. Object is on the left, messages on the right. Messages can be without arguments:

\begin{lstlisting}
obj message
\end{lstlisting}

(to the \textit{obj} send a \textit{message}), or with arguments:

\begin{lstlisting}
obj message: argument
\end{lstlisting}

(to the \textit{obj} send a \textit{message} with \textit{argument} as value).




\columnbreak
{\small
\subsection{Special slots}

\subsubsection{\texttt{self} slot}
Every object has automatically created slot named \texttt{self}, pointing to object itself. Unlike other languages, \texttt{self} may be omitted, in \textit{message sends}, so

\begin{lstlisting}
self message
\end{lstlisting}

is same thing as

\begin{lstlisting}
message
\end{lstlisting}



\subsubsection{Annotation slot}

Additional informations may be provided in the \textit{annotation slot}:

\begin{lstlisting}
(| {} = 'Annotation string.' |)
\end{lstlisting}

or

\begin{lstlisting}
(| { 'Annotation string.' slot. another. } |)
\end{lstlisting}



\subsubsection{\texttt{parent} slots}
Every object may also contain one or more parent slots, which are slots whose names end in a star (\texttt{*}). Lookups unresolved in the receiving object are delegated to these parent slots. This is used to implement inheritance (often with a single parent slot named \texttt{parent*}) and also mixins, and namespaces.
}



\subsection{Method slots}

\textit{Method slots} is the storage for the code in objects.



\subsubsection{Unary slots}
Code \textit{slots} without arguments are called \textit{unary}.

\begin{lstlisting}
(| first = (printLine) |)
\end{lstlisting}

This \textit{method slot} can be invoked by sending \textit{unary} message:

\begin{lstlisting}
obj first
\end{lstlisting}



\subsubsection{Binary slots}

Slots with one argument is called \textit{binary}:

\begin{lstlisting}
(| first: = (| :arg | arg printLine) |)
\end{lstlisting}

Which has shorter equivalent:

\begin{lstlisting}
(| first: arg = (arg printLine) |)
\end{lstlisting}

Invocation is possible using the \textit{binary message}:

\begin{lstlisting}
obj first: 1
\end{lstlisting}

\textit{Binary slots} are also used as operators (one or more characters from \texttt{!@\#\$\%\^{}\&*-+=\textasciitilde/?\textless\textgreater,;|'\textbackslash{}} set).



\subsubsection{Keyword slots}

It is also possible to create \textit{keyword} slots with multiple arguments:

\begin{lstlisting}
(| first: x Second: y = (x + y printLine.) |)
\end{lstlisting}

Notice the upper-case \textit{S} in \texttt{Second}. Invocation is done via \textit{keyword message}:

\begin{lstlisting}
obj first: 3 Second: 5
\end{lstlisting}



\subsubsection{Priorities}

Constant definition \textgreater{} Unary \textgreater{} Binary \textgreater{} Keyword messages. Constant = self \textbar{} number \textbar{} string \textbar{} object.




\subsection{Block objects \textit{(closures)}}

\textit{Block objects} are Self closures, which means, that \texttt{self} slot is not present. Unresolved lookups are delegated via \texttt{parent*} to namespace surrounding the \textit{block object} at the time of creation. Syntax is same as with objects, but square brackets are used:

\begin{lstlisting}
[| slot1. slot2 | 'code']
\end{lstlisting}


Another difference is support of \textit{non-local return statement} using \texttt{\^} preceding the returned value. This returns the value not just from the \textit{closure}, but also from the enclosing namespace.

\begin{lstlisting}
(|
    x = 6.
    non_zero = (x > 5 ifTrue: [ ^ 'ok' ]
                         False: [ ^ 'nope' ])
|)
\end{lstlisting}

\^{} doesn't just return from the closure in \texttt{[]}, but also from the surrounding \textit{method object}. \texttt{non\_zero} message thus returns \texttt{'ok'}.



\subsubsection{Block messages}

\textit{Block objects} take by default message \texttt{value}, which evaluates the code and returns result of last statement.

If the \textit{block object} takes (multiple) parameters, they may be supplied using \texttt{value:\ With:} message pattern:

\begin{lstlisting}
block_obj value: x With: y .. With: z
\end{lstlisting}

Notice the upper-case first letters in \texttt{With:} message. This tells the Self, that it's still part of one message with multiple arguments. In case of blocks, \textbf{unrequested arguments are ignored}, so you may use unlimited number of \texttt{With:}.




%%%%%%%%%%%%%%%%%%%%%%%%%%%%%%%%%%%%%%%%%%%%%%%%%%%%%%%%%%%%%%%%%%%%%%%%%%%%%%%
\pagebreak %%%%%%%%%%%%%%%%%%%%%%%%%%%%%%%%%%%%%%%%%%%%%%%%%%%%%%%%%%%%%%%%%%%%
\section{Namespaces}

\subsection{Traits}
Objects with shared behaviour, which is used by pointing \texttt{parent*} slots to them. This is analogical to \textit{sub-classing} in other languages.



\subsection{Mixins}
Mixins are small, parentless bundles of behavior designed to be \textit{mixed into} some other object. Approximate analogy to other languages would be \textit{interface with (partial) implementation}.



\subsection{Globals}
Prototype objects and \textit{oddballs} (unique singletons) like \texttt{true}, \texttt{false}, \texttt{nil}..




\section{Resends}
\textit{Resends} allows you to delegate the messages to the parent branches of OOP tree. \textit{Resend messages} are equivalent of \textit{super} calls in other languages.

Syntactically, \textit{resends} are implemented using \texttt{resend.} prefix for resent messages:

\begin{lstlisting}
resend.unary
resend.+ 1
resend.keyword: 1 Another: 2
\end{lstlisting}

You may also use \textit{directed resends}, targeting specific parent:

\begin{lstlisting}
intParent.+ 1
otherParent.keyword: 1 Another: 2
\end{lstlisting}




\section{Mirrors}
\textit{Mirrors} provide Self with introspection capabilities.

\textit{Mirror} can be created by sending \texttt{reflect:\ x} message to any \texttt{defaultBehavior} instance. The message will return \textit{dictionary-like} \textit{mirror} object.

\begin{lstlisting}
defaultBehavior reflect: 1
\end{lstlisting}

\textit{Mirror} presents you with sort of introspection layer usable for structural changes in \textit{reflectee} (original object \texttt{x}) by manipulating the \textit{dictionary-like mirror} object.

This may be used for examination, addition or removal of the \textit{slots} of the \textit{reflectee}.




\vfill
\columnbreak
\section{Collections}
Various containers for data are implemented as \textit{key:\ val} storages. Even lists use this convention (elements are used both as \textit{key} and \textit{value}).

Self offers a rich variations of \textit{Sets}, \textit{Dictionaries} and \textit{Trees}:

\vspace*{0.4cm}
% http://texblog.org/2012/08/07/semi-automatic-directory-tree-in-latex/
\dirtree{%
 .1 traits \textbf{collection}.
 .2 traits \textbf{abstractSetOrDictionary}.
 .3 traits \textbf{abstractSet}.
 .4 traits \textbf{universalSet}.
 .5 \textbf{universalSet}.
 .4 traits \textbf{hashTableSet}.
 .5 \textbf{identitySet} parent.
 .5 \textbf{customizableSet} parent.
 .5 \textbf{reflectiveIdentitySet} parent.
 .5 traits \textbf{set}.
 .5 traits \textbf{sharedSet}.
 .3 traits \textbf{abstractDictionary}.
 .4 traits \textbf{orderedDictionary}.
 .4 traits \textbf{universalDictionary}.
 .5 \textbf{universalDictionary}.
 .4 traits \textbf{hashTableDictionary}.
 .5 \textbf{identityDictionary} parent.
 .5 \textbf{customizableDictionary} parent.
 .5 \textbf{reflectiveIdentityDictionary} parent.
 .5 traits \textbf{dictionary}.
 .5 traits \textbf{sharedDictionary}.
 .2 traits \textbf{tree}.
 .3 traits \textbf{emptyTrees} abstract.
 .4 traits \textbf{emptyTrees bag}.
 .4 traits \textbf{emptyTrees set}.
 .3 traits \textbf{treeNodes} abstract.
 .4 traits \textbf{treeNodes bag}.
 .4 traits \textbf{treeNodes set}.
}

\vspace*{0.4cm}
\textit{Sets} behave like mathematical sets - unordered unique collection of values. \textit{Dictionaries} work as \textit{key: val} storages and are implemented using hashmaps.

\textit{Trees} are different implementations of \textit{dictionaries} using \textit{unbalanced binary trees}.

Note: If the elements are added in sorted order, \textit{trees} may degenerate into lists, which may result in really bad performance.




\columnbreak

There is also variety of \textit{Lists}, \textit{Vectors}, \textit{Strings} and \textit{Queues}:

\vspace*{0.4cm}
\dirtree{%
 .1 traits \textbf{collection}.
 .2 traits \textbf{path}.
 .2 traits \textbf{sharedQueue}.
 .2 traits \textbf{priorityQueue}.
 .2 traits \textbf{list}.
 .3 traits \textbf{orderedSet}.
 .3 \textbf{sortedList} parent.
 .4 \textbf{sortedListSet} parent.
 .2 traits \textbf{indexable}.
 .3 traits \textbf{mutableIndexable}.
 .4 traits \textbf{sequence}.
 .5 traits \textbf{sortedSequence}.
 .4 traits \textbf{vector}.
 .5 \textbf{vector}.
 .4 traits \textbf{byteVector}.
 .5 \textbf{byteVector}.
 .5 traits \textbf{int32or64}.
 .6 traits \textbf{int64}.
 .6 traits \textbf{int32}.
 .5 traits \textbf{string}.
 .6 traits \textbf{immutableString}.
 .7 traits \textbf{canonicalString}.
 .6 traits \textbf{mutableString}.
}

\vspace*{0.4cm}

Collections have rich message protocol, allowing various operations. Most important are:


\vspace*{0.2cm}
\small{\begin{tabular}{ p{50pt} p{180pt} l l }
Message & Description
\\\hline\hline

%%%%%%%%%%%%%%%%%%%%%%%%%%%%%%%%%%%%%%%%%%%%%%%%%%%%%%%%%%%%%%%%%%%%%%%%%%%%%%%
\texttt{at:}
&
Get item at position / key.
\\\hline %%%%%%%%%%%%%%%%%%%%%%%%%%%%%%%%%%%%%%%%%%%%%%%%%%%%%%%%%%%%%%%%%%%%%%

%%%%%%%%%%%%%%%%%%%%%%%%%%%%%%%%%%%%%%%%%%%%%%%%%%%%%%%%%%%%%%%%%%%%%%%%%%%%%%%
\texttt{at: Put:}
&
Put item to position / key.
\\\hline %%%%%%%%%%%%%%%%%%%%%%%%%%%%%%%%%%%%%%%%%%%%%%%%%%%%%%%%%%%%%%%%%%%%%%

%%%%%%%%%%%%%%%%%%%%%%%%%%%%%%%%%%%%%%%%%%%%%%%%%%%%%%%%%%%%%%%%%%%%%%%%%%%%%%%
\texttt{add:}
&
Add item (to the end in ordered) collections.
\\\hline %%%%%%%%%%%%%%%%%%%%%%%%%%%%%%%%%%%%%%%%%%%%%%%%%%%%%%%%%%%%%%%%%%%%%%

%%%%%%%%%%%%%%%%%%%%%%%%%%%%%%%%%%%%%%%%%%%%%%%%%%%%%%%%%%%%%%%%%%%%%%%%%%%%%%%
\texttt{addAll:}
&
Add all items to (end of) collection.
\\\hline %%%%%%%%%%%%%%%%%%%%%%%%%%%%%%%%%%%%%%%%%%%%%%%%%%%%%%%%%%%%%%%%%%%%%%

%%%%%%%%%%%%%%%%%%%%%%%%%%%%%%%%%%%%%%%%%%%%%%%%%%%%%%%%%%%%%%%%%%%%%%%%%%%%%%%
\texttt{do:}
&
Iterate over collections.
\\ %%%%%%%%%%%%%%%%%%%%%%%%%%%%%%%%%%%%%%%%%%%%%%%%%%%%%%%%%%%%%%%%%%%%%%%%%%%%
\end{tabular}}



\subsection{Collector}
\textit{Collector} is special kind of object created using \texttt{\&} operator. \textit{Collector} is not a collection, but can be converted to one.

Main reason to use it is the \texttt{\&} operator, which may simplify the syntax required to create such collection.

\nointerlineskip\begin{lstlisting}
(1 & '+' & 2) asList
\end{lstlisting}\nointerlineskip



\subsection{Point}
Another kind of container often used with \textit{collections} is \texttt{point} and naturally the \texttt{rectangle} made of two points.




\vfill
\columnbreak
\section{Control sequences}
As usual in \textit{Smalltalk-like} languages, \textit{control sequences} are implemented using message sends combined with block.



\subsection{Conditionals}
\textit{If condition} works by using messages defined in \texttt{boolean} (or \texttt{boolean traits}) objects:

\vspace*{0.2cm}

\small{\begin{tabular}{ p{70pt} p{140pt} l l }
Message & Description
\\\hline\hline

%%%%%%%%%%%%%%%%%%%%%%%%%%%%%%%%%%%%%%%%%%%%%%%%%%%%%%%%%%%%%%%%%%%%%%%%%%%%%%%
\texttt{obj ifTrue: b}
&
Execute \texttt{b} if the \textit{obj} is \texttt{true}.
\\\hline %%%%%%%%%%%%%%%%%%%%%%%%%%%%%%%%%%%%%%%%%%%%%%%%%%%%%%%%%%%%%%%%%%%%%%

%%%%%%%%%%%%%%%%%%%%%%%%%%%%%%%%%%%%%%%%%%%%%%%%%%%%%%%%%%%%%%%%%%%%%%%%%%%%%%%
\texttt{obj\ ifFalse:\ b}
&
Execute \texttt{b} if the \textit{obj} is \texttt{false}.
\\\hline %%%%%%%%%%%%%%%%%%%%%%%%%%%%%%%%%%%%%%%%%%%%%%%%%%%%%%%%%%%%%%%%%%%%%%

%%%%%%%%%%%%%%%%%%%%%%%%%%%%%%%%%%%%%%%%%%%%%%%%%%%%%%%%%%%%%%%%%%%%%%%%%%%%%%%
\nointerlineskip
\begin{lstlisting}[aboveskip=0pt,belowskip=-0.8 \baselineskip]
obj ifTrue: b1
      False: b2
\end{lstlisting}
&
\textit{b1} is executed, when \textit{obj} evaluates to \texttt{true}.
If not, (optional) \textit{b2} block is used.
\\\hline %%%%%%%%%%%%%%%%%%%%%%%%%%%%%%%%%%%%%%%%%%%%%%%%%%%%%%%%%%%%%%%%%%%%%%

%%%%%%%%%%%%%%%%%%%%%%%%%%%%%%%%%%%%%%%%%%%%%%%%%%%%%%%%%%%%%%%%%%%%%%%%%%%%%%%
\nointerlineskip
\begin{lstlisting}[aboveskip=-8pt,belowskip=-0.8 \baselineskip]
obj ifFalse: b1
        True: b2
\end{lstlisting}
&
Opposite of previous.
\\ %%%%%%%%%%%%%%%%%%%%%%%%%%%%%%%%%%%%%%%%%%%%%%%%%%%%%%%%%%%%%%%%%%%%%%%%%%%%
\end{tabular}}




\subsection{Loops}
Looping is implemented by sending various loop messages to \textit{block objects}:

\begin{lstlisting}
[ ... ] loop
\end{lstlisting}
Loop over the \textit{block} indefinitely.

\vspace*{0.2cm}
\hrule

\begin{lstlisting}
[ proceed ] whileTrue: [ ... ]
\end{lstlisting}
Loop while \texttt{proceed} is \texttt{true}.

\vspace*{0.2cm}
\hrule

\begin{lstlisting}
[ quit ] whileFalse: [ ... ]
\end{lstlisting}
Loop while \texttt{quit} is \texttt{false}.

\vspace*{0.2cm}
\hrule

\begin{lstlisting}
[ ... ] untilTrue: [ quit ]
\end{lstlisting}
Loop until \texttt{quit} is \texttt{true}. Loop at least once.

\vspace*{0.2cm}
\hrule

\begin{lstlisting}
[ ... ] untilFalse: [ proceed ]
\end{lstlisting}
Loop until \texttt{proceed} is \texttt{false}. Loop at least once.

\vspace*{0.2cm}
\hrule

\begin{lstlisting}
[| :exit | ... cond ifTrue: exit ... ] loopExit
\end{lstlisting}
Loop until the \texttt{exit} parameter is not evaluated.

\vspace*{0.2cm}
\hrule

\begin{lstlisting}
[
    | :exit |
    ...
    cond ifTrue: [ exit value: expr ]
] loopExitValue
\end{lstlisting}
Loop until the \texttt{exit} parameter is not evaluated. Allows to return the \texttt{value} with the exit from the loop.




\subsubsection{integerIteration loops}
There is also loops defined as message sends to integers:

\begin{lstlisting}
numObj do: block
\end{lstlisting}

Do the \texttt{block} \textit{numObj} times.

\vspace*{0.2cm}
\hrule

\begin{lstlisting}
numObj to: end Do: block
\end{lstlisting}

Do the \texttt{block} each time counting from \textit{numObj} to \texttt{end}. For example \texttt{5\ to:\ 8\ Do:\ [|\ :i\ |\ i\ print]} will print \texttt{5678}.

Following messages are all variations of this message:

\begin{lstlisting}
numObj to: By: Do:
numObj to: ByNegative: Do:
numObj to: ByPositive: Do:
numObj upTo: Do:
numObj upTo: By: Do:
numObj downTo: Do:
numObj downTo: By: Do:
\end{lstlisting}




\section{Useful bits}
\subsection{copy message}
Often, you will need to create new instance of some object. What is in other languages implemented by calling \texttt{new}, or other processes calling the object builders is in prototype languages usually done by copying the object into new instance.

In Self, you may do this simply by sending the \texttt{copy} message:

\begin{lstlisting}
obj copy
\end{lstlisting}



\subsection{Interpreter parameters}
When you run the Self interpreter on the \textit{commandline}, it expects that you specify the path to the \textit{World snapshot}. That may be supplied using \texttt{-s} parameter.

Self \textit{worlds} are images for the virtual machine. You may use the official, but there is nothing that prevents you to create your own, with or without Graphical User Interface.



\subsection{Self's REPL}
When you run your world, it is easy to forget, that Self's interpreter also provides you with Read Eval Print Loop in the terminal. This allows you to send messages like if you were in the Shell in GUI. For example, if no windows opens, or the \textit{desktop} crashes, you may open it by sending:

\begin{lstlisting}
desktop open
\end{lstlisting}

\texttt{CTRL+c} will bring the scheduler, where you may kill processes.

You may also \href{http://handbook.selflanguage.org/4.5/textdebug.html}{debug} processes using \texttt{debugger\ attach:\ N}.




\vfill
\columnbreak
\section{Additional informations}



\subsection{Download}
Newest Self may be downloaded from the official web:

\begin{itemize}[noitemsep]
\item \url{http://selflanguage.org}
\end{itemize}

Source code may be obtained from the GitHub:

\begin{itemize}[noitemsep]
\item \url{https://github.com/russellallen/self}
\end{itemize}



\subsection{Manual and tutorial}
There is good manual called Self handbook:

\begin{itemize}[noitemsep]
\item \href{http://handbook.selflanguage.org/4.5/}{Self Handbook} (online)
\item \href{https://dl.dropboxusercontent.com/u/11891854/tmp/ebooks/Self/SelfHandbook.pdf}{Self Handbook} (pdf)
\item \href{https://dl.dropboxusercontent.com/u/11891854/tmp/ebooks/Self/SelfHandbook.epub}{Self Handbook} (ePub)
\end{itemize}

This \textit{Cheat sheet} was created with high inspiration from this book.

If one of the links doesn't work, you may always build it for yourself from sources (\texttt{/docs/handbook}) using Sphinx (\texttt{make} \texttt{html} / \texttt{pdf} / \texttt{epub}).

There is also slightly outdated Self tutorial (\texttt{/docs/tutorial}) called \href{http://kitakitsune.org/bhole/self_tutorial/}{\footnotesize Prototype-Based Application Construction Using SELF 4.0}.



\subsection{Other writings}

There is blog:

\begin{itemize}[noitemsep]
\item \url{http://blog.selflanguage.org}
\end{itemize}

and also a lot of academic papers:

\begin{itemize}[noitemsep]
\item \url{http://bibliography.selflanguage.org}
\end{itemize}



\subsection{Community}
There is not-quite dead IRC channel \texttt{\#self-lang} at \texttt{irc://freenode.org:6667}.

There is also forum at:

\begin{itemize}[noitemsep]
\item \url{http://forum.selflanguage.org/}
\end{itemize}

If the forum doesn't work, don't worry, it is just frontend for mail conference \href{https://groups.yahoo.com/neo/groups/self-interest/info}{self-interest-subscribe@yahoogroups.com}.



\subsection{Disclaimer}
I did this as my notes while learning Self. That means, that there will be typos and bugs. If you find any, please send pull request.

Self may seem dead, but it is not. There is not much active people, but that only means, that you can make difference.




% Footer
\vspace*{0.4cm}
\hrule
\smallskip
{\small
This file may be freely distributed under
the terms of the MIT license ---
Copyleft \textcopyleft\ \thedate{} \href{http://kitakitsune.org}{\theauthor}. Version 1.1 (07.03.2016).

\url{https://github.com/Bystroushaak/SelfCheatSheet}
}

\end{multicols*}
\end{document}
